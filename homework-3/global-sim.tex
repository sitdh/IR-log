%!TEX TS-program = xelatex
%!TEX encoding = UTF-8 Unicode
\documentclass[11pt,a4paper]{report}
\usepackage{fontspec,xltxtra,xunicode}
\usepackage[top=1.5in,bottom=1in,left=1.5in,right=1.5in]{geometry}
\usepackage{polyglossia}
\newfontfamily\thaifont{TH Sarabun New}
\setmainfont{TH Sarabun New}
\setdefaultlanguage{thai}
\XeTeXlinebreaklocale 'th'
\usepackage{scrextend}
% \changefontsizes[20pt]{16pt}
\XeTeXlinebreakskip = 0pt plus 1pt
\defaultfontfeatures{Scale=1.23}
\renewcommand{\baselinestretch}{1.2}

\usepackage{fancyhdr}
\pagestyle{fancy}
\lhead{\bf{การบ้านครั้งที่ 3: คำนวณหา Global Similarity Thesaurus}}
\rhead{2110xxx: Information Store and Retrieval}
\cfoot{สิทธิพงษ์ เหล่าโก้ก \\
5870972621 วิศวกรรมซอฟต์แวร์ ภาคนอกเวลาราชการ}

\begin{document}
    \emph{\bf{โจทย์}}
        จงปรับคำค้นตามวิธีการคำนวณแบบ Global Similarity Thesaurus โดยใช้ข้อคำถาม (query) จำนวน 2 ข้อคำถาม คำ (term) ขนาด 10 คำ เพื่อใช้สืบค้นเอกสาร จากจำนวนเอกสารทั้งสิ้น 10 ชิ้น 

    \emph{\bf{วิธีทำ}}  

    กำหนดให้ ข้อคำถาม คือ \emph{"Python programing"} ($q_1$) และ \emph{"Data science"} ($q_2$) โดยจะสืบค้นจากคำอธิบายรายวิชาของวิชา ($D_j$) ที่ได้จากเว็บไซต์ coursera.org, edx.org และ udemy.com ซึ่งประกอบไปด้วยรายวิชาทั้งสิ้น 10 วิชา ($|D_j|$) ดังนี้
    \begin{itemize}
        \item ddd
        \item ddd
        \item ddd
        \item ddd
        \item ddd
        \item ddd
        \item ddd
        \item ddd
        \item ddd
        \item ddd
    \end{itemize}
        
\end{document}

