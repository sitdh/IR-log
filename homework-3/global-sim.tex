%!TEX TS-program = xelatex
%!TEX encoding = UTF-8 Unicode
\documentclass[11pt,a4paper]{article}
\usepackage{fontspec,xltxtra,xunicode}
\usepackage[top=1.5in,bottom=1in,left=1.5in,right=1.5in]{geometry}
\usepackage{polyglossia}
\newfontfamily\thaifont{TH Sarabun New}
\setmainfont{TH Sarabun New}
\setdefaultlanguage{thai}
\XeTeXlinebreaklocale 'th'
\usepackage{scrextend}
% \changefontsizes[20pt]{16pt}
\XeTeXlinebreakskip = 0pt plus 1pt
\defaultfontfeatures{Scale=1.23}
\renewcommand{\baselinestretch}{1.2}

\newcommand{\numberOfDocument}{5}
\newcommand{\numberOfQuery}{2}
\newcommand{\numberOfTerm}{2}
\newcommand{\numberOfCollection}{10}

\newcommand{\fir}{Python}
\newcommand{\seco}{course}
\newcommand{\thi}{data}
\newcommand{\fou}{learn}
\newcommand{\fif}{advanced}
\newcommand{\six}{science}
\newcommand{\sev}{spatial}
\newcommand{\eig}{money}
\newcommand{\nin}{code}
\newcommand{\ten}{programing}

\usepackage{amsmath}
\usepackage{csvsimple}
\usepackage{lipsum}

\usepackage{fancyhdr}
\pagestyle{fancy}
\lhead{\bf{การบ้านครั้งที่ 3: คำนวณหา Global Similarity Thesaurus}}
\rhead{2110xxx: Information Store and Retrieval}
\cfoot{สิทธิพงษ์ เหล่าโก้ก \\
5870972621 วิศวกรรมซอฟต์แวร์ ภาคนอกเวลาราชการ}

\begin{document}
    \emph{\bf{โจทย์}}
        จงปรับคำค้นตามวิธีการคำนวณแบบ Global Similarity Thesaurus โดยใช้ข้อคำถาม (query) จำนวน \numberOfQuery\ ข้อคำถาม คำ (term) ขนาด \numberOfTerm\ คำ เพื่อใช้สืบค้นเอกสาร จากจำนวนเอกสารทั้งสิ้น \numberOfDocument\ ชิ้น 

    \emph{\bf{วิธีทำ}}  

    \section{จัดเตรียมข้อมูล}

    กำหนดให้ ข้อคำถามแรก ($q_1$) คือ \emph{"Python programing"} และ ข้อคำถามที่สอง ($q_2$) คือ \emph{"Data science"} โดยจะสืบค้นจากคำอธิบายรายวิชาของวิชา ($D$) ที่ได้จากเว็บไซต์ coursera.org, edx.org และ udemy.com ซึ่งประกอบไปด้วยรายวิชาทั้งสิ้น \numberOfDocument\ วิชา ($|D|$) ดังรายการด้านล่าง

    \begin{itemize}
        \item[$d_1$:] {\bf Python for Geospatial}  \\
            \input{documents/raw/documents-1.txt}

        \item[$d_2$:] {\bf Data science with Pandas in Python 3}  \\
            \input{documents/raw/documents-2.txt}

        \item[$d_3$:] {\bf Complete Python Bootcamp} \\
            \input{documents/raw/documents-3.txt}

        \item[$d_4$:] {\bf Using Python for Research} \\
            \input{documents/raw/documents-4.txt}

        \item[$d_5$:] {\bf Programming with Python for Data Science} \\
            \input{documents/raw/documents-5.txt}

        % \item[$d_6$:] Introduction to Python for Data Science
        % \item[$d_7$:] Learn to Program Using Python
        % \item[$d_8$:] Introduction to Data Science in Python
        % \item[$d_9$:] Using Python to Access Web Data
        % \item[$d_{10}$:] Introduction to Data Science in Python
    \end{itemize}

    จากนั้นจึงดึงคำที่เป็น Stop word ออกจากเอกสาร โดยใช้ชุดคำสั่งที่ชื่อว่า nltk (http://nltk.org) ซึ่งเป็นชุดคำสั่งที่ใช้เพื่อการประมวลผลภาษาธรรมชาติ (Natural language processing) เพื่อให้เอกสารที่ได้นั้นมีความสมบูรณ์มากยิ่งขึ้น

    \clearpage
    \section{หาค่าความถี่คำที่สนใจ}
    ในขั้นตอนนี้จะหาความถี่ของคำที่สนใจทั้งหมด \numberOfCollection คำ ($|V|$) จากกลุ่มเอกสาร โดยให้ $V$ แทนกลุ่มของคำศัพท์ที่สนใจ จะได้ว่า  \\ 
    {V = \{"{\fir}",\ "{\seco}",\ "{\thi}",\ "{\fou}",\ "{\fif}",\ "{\six}",\ "{\sev}",\ "{\eig}",\ "{\nin}",\ "{\ten}"\}

    โดยพบความถี่ของคำแยกตามเอกสาร ดังนี้
    \begin{table}[ht!]
        \centering
        \caption{สรุปความถี่ของคำที่ปรากฎในเอกสาร}
        \label{tab:freq}
        \begin{tabular}{ p{3cm}ccccc}
            \hline
            คำ ($k_i$)      & $d_1$ &  $d_2$    &  $d_3$    &  $d_4$    &  $d_5$ \\
            \hline \hline
            \fir            & 7     &  2        &  7        &  7        &  3  \\
            \seco           & 4     &  5        &  8        &  2        &  4  \\
            \thi            & 1     &  5        &  0        &  0        &  7  \\
            \fou            & 1     &  3        &  3        &  1        &  0  \\
            \fif            & 0     &  1        &  1        &  1        &  0  \\
            \six            & 0     &  3        &  0        &  0        &  0  \\
            \sev            & 3     &  0        &  0        &  0        &  0  \\
            \eig            & 0     &  1        &  2        &  0        &  0  \\
            \nin            & 3     &  0        &  1        &  0        &  0 \\
            \ten            & 0     &  0        &  1        &  0        &  2  \\
            \hline 
            $max(f_{i,j})$  & 7     &  5        &  8        &  7        &  7  \\
            \hline
        \end{tabular}
    \end{table}

    \clearpage
    \section{คำนวณค่า Inverse term frequency}
    จากนั้นจึงหากค่า Inverse term frequency ของแต่ละคำใน\tablename\ \ref{tab:freq} โดยแยกตามเอกสาร $d_j$ ดังสูตร

    \begin{equation} 
        \label{eq:itf}
        itf_j = \log{\frac{t}{t_j}}
    \end{equation}

    \begin{table}[ht!]
        \begin{tabular}{p{2cm}l}
            เมื่อ & $t$ คือ จำนวนคำในกลุ่มที่กำหนด ในที่นี้คือ \numberOfCollection\ \\
                & $t_j$ คือ จำนวนของคำศัพท์ที่ไม่ซ้ำกัน และพบในเอกสาร $d_j$ \\
                & $itf_j$ คือ ค่า Inverse term frequency ที่คำนวณได้ \\
        \end{tabular}
    \end{table}
    {\bf หมายเหตุ:} ฐานของ $\log$ ในสมการที่ \ref{eq:itf} มีค่าเป็น 2  \\\\

    โดยที่ $t$ มีค่าเป็น \numberOfCollection และ ค่า $t_1$, $t_2$, $t_3$, $t_4$, $t_5$ มีค่าเป็น 
    \begin{table}[ht!]
        \centering
        \caption{ความถี่ของคำศัพท์ที่ไม่ซ้ำกันและพบในเอกสาร}
        \label{tab:keywordfreq}
        \begin{tabular}{|l||ccccc|}
            \hline
            $d_j$ & $d_1$ & $d_2$ & $d_3$ & $d_4$ & $d_5$ \\
            \hline
            $t_j$ & 6     & 7     & 7     & 4     & 4 \\
            \hline
        \end{tabular}
    \end{table}

    ยกตัวอย่างเช่น ต้องการคำนวณค่า $itf_1$ ของคำว่า $Python$ ในเอกสาร $d_1$ โดยอ้างจากสมการ \ref{eq:itf} จะได้ว่า
    \begin{equation*}
        \begin{aligned}
            itf_j &= \log{\frac{t}{t_j}} \\
            itf_i &= \log{\frac{10}{6}} \\
                  &= \log{1.6667} \\
                  &= 0.7370
        \end{aligned}
    \end{equation*}

    ดังนั้น สามารถคำนวณค่า $itf$ ของคำอื่นๆ ในแต่ละเอกสารได้ดังตารางด้านล่าง
    % \begin{table}[ht!]
    %     \centering
    %     \caption{ค่า itf ของคำในแต่ละเอกสาร}
    %     \label{tab:itf}
    %     \begin{tabular}{l|cc}
    %         $k_j$   & $itf_j$   & $itf_j^2$\\
    %         \hline \hline
    %         $k_1$   & 0.7370    & 0.5432 \\ 
    %         $k_2$   & 0.5146    & 0.0239 \\
    %         $k_3$   & 0.5146    & 0.0239 \\
    %         $k_4$   & 1.3219    & 1.7474 \\
    %         $k_5$   & 0.3219    & 0.1036 \\
    %     \end{tabular}
    % \end{table}

    \begin{table}[ht!]
        \centering
        \caption{ค่าน้ำหนักของคำแยกตามเอกสาร คำนวณโดยใช้สมการ}
        \label{tab:weight}
        \begin{tabular}{lccccc}\hline%
                        & $d_1$ & $d_2$ & $d_3$     & $d_4$     & $d_5$
            \\\hline \hline
            \csvreader[head to column names]{itf-cal.csv}{}%
            {$itf_j$  & \one  & \two  & \three    & \four     & \five}
            \\\hline 
        \end{tabular}
    \end{table}

    \clearpage
    \section{คำนวณนำหนักของคำ}
    เพื่อหาเวกเตอร์ $\overrightarrow{k}_i$ ที่อยู่ในรูปดังสมการ \ref{eq:kvector} ซึ่งประกอบไปด้วย น้ำหนักของคำที่พบในเอกสาร ($w_{i,j}$) ซึ่งน้ำหนักของคำนี้สามารถคำนวณได้ด้วยสมการที่ \ref{eq:weightcal} โดยพิจารณาน้ำหนักของคำในแต่ละเอกสาร

    \begin{equation}
        \label{eq:kvector}
        \overrightarrow{k}_i = (w_{i,1}, w_{i,2}, w_{i,3}, ..., w_{i,N})
    \end{equation}

    \begin{equation}
        \label{eq:weightcal}
        w_{i,j} = \frac{(0.5 + 0.5\frac{f_{i,j}}{max_i(f_{i,j})}) itf_{j}}
                       {\sqrt{\sum_{i=1}^{N}({0.5 + 0.5\frac{f_{i,j}}{max_l(f_{i,l})})^2 itf_j^2}}}
    \end{equation}

    \begin{table}[ht!]
        \begin{tabular}{p{1cm}l}
            เมื่อ & $w_{i,j}$ คือ น้ำหนักของ $k_i$ ที่พบภายใน $d_j$ ใดๆ \\
                & $\overrightarrow{k}_i$ คือ เวกเตอร์ของ $k_j$ ที่ประกอบขึ้นจากน้ำหนักของคำ $k_j$ ใดๆ ในทุกๆ เอกสาร \\
                & $f_{i,j}$ คือ ความถี่ของ $k_i$ ใดๆ ในเอกสาร $d_j$ \\
                & $f_{i,l}$ คือ ความถี่ของ $k_i$ ใดๆ ในเอกสารทั้งหมด \\
        \end{tabular}
    \end{table}

    พิจารณาพจน์ $(0.5 + 0.5\frac{f_{i,j}}{max(f_{i,j})})$ โดยกำหนดให้ 
    \begin{equation}
        \label{eq:partone}
        \begin{aligned}
            w_{i,j}' = 0.5 + 0.5\frac{f_{i,j}}{max(f_{i,j})} itf_j
        \end{aligned}
    \end{equation}

    ตัวอย่างการคำนวณ เพื่อหาค่า $w_{i,j}'$ จาก (\ref{eq:partone}) จะได้ว่า
    \begin{equation*}
        \begin{aligned}
            w_{i,j}'    &= 0.5 + 0.5\frac{f_{i,j}}{max(f_{i,j})} itf_j \\
             w_{1,1}'   &= 0.5 + 0.5(\frac{7}{6}) itf_1 \\
                        &= 0.5 + 0.5(\frac{7}{6}) 0.7370 \\
                        &= 0.7370 
        \end{aligned}
    \end{equation*}

    \begin{table}[ht!]
        \centering
        \caption{ส่วนประกอบสมาการ เพื่อหาค่า $w_{i,j}'$}
        \label{tab:partone}
        \begin{tabular}{lccccc}\hline%
            คำ ($k_i$) & $itf_1$   & $itf_2$   & $itf_3$   & $itf_4$   & $itf_5$
            \\\hline \hline
            \csvreader[head to column names]{above-eq.csv}{}%
            {\\$k_{\thecsvrow}$ & \one & \two & \three & \four & \five}
            \\\hline 
        \end{tabular}
    \end{table}

    % \begin{table}[ht!]
    %     \centering
    %     \caption{ค่าน้ำหนักของคำแยกตามเอกสาร คำนวณโดยใช้สมการ \ref{eq:weightcal}}
    %     \label{tab:weight}
    %     \begin{tabular}{lccccc}\hline%
    %         คำ ($k_i$) & $itf_1$   & $itf_2$   & $itf_3$   & $itf_4$   & $itf_5$
    %         \\\hline \hline
    %         \csvreader[head to column names]{itf-cal.csv}{}%
    %         {\\\term   & \one & \two & \three & \four & \five}
    %         \\\hline 
    %     \end{tabular}
    % \end{table}

    ดังนั้น สามารถใช้ข้อมูลจาก \tablename\ \ref{tab:freq} และ\tablename\ \ref{tab:itf} เข้าร่วมในการคำนวณตามสมการที่ \ref{eq:weightcal} ตัวอย่างเช่น

    \begin{equation*}
        \begin{aligned}
            w_{i,j} &= \frac{(0.5 + 0.5\frac{f_{i,j}}{max_i(f_{i,j})}) itf_{j}}
                            {\sqrt{\sum_{i=1}^{N}({0.5 + 0.5\frac{f_{i,j}}{max_l(f_{i,l})})^2 itf_l^2}}} \\
            w_{1,1} &= \frac{(0.5 + 0.5(\frac{7}{7})0.5146}
                            {
                                \sqrt{
                                    (0.5 + 0.5 (\frac{6}{7})^2) (0.7370)^2)
                                    + (0.5 + 0.5 (\frac{7}{7})^2) (0.5146)^2)
                                    + (0.5 + 0.5 (\frac{7}{7})^2) (0.5146)^2)
                                    + (0.5 + 0.5 (\frac{4}{7})^2) (1.3219)^2)
                                    + (0.5 + 0.5 (\frac{4}{7})^2) (0.3219)^2)
                                }
                            } \\
        \end{aligned}
    \end{equation*}

    % \begin{table}[ht!]
    %     \centering
    %     \caption{ค่าน้ำหนักของคำแยกตามเอกสาร คำนวณโดยใช้สมการ \ref{eq:weightcal}}
    %     \label{tab:weight}
    %     \begin{tabular}{lccccc}\hline%
    %         คำ ($k_i$) & $itf_1$   & $itf_2$   & $itf_3$   & $itf_4$   & $itf_5$
    %         \\\hline \hline
    %         \csvreader[head to column names]{itf-cal.csv}{}%
    %         {\\\term   & \one & \two & \three & \four & \five}
    %         \\\hline 
    %     \end{tabular}
    % \end{table}


\end{document}

