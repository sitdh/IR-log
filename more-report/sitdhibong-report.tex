%!TEX TS-program = xelatex
%!TEX encoding = UTF-8 Unicode
\documentclass[11pt,a4paper]{article}
\usepackage{fontspec,xltxtra,xunicode}
\usepackage[top=1.5in,bottom=1in,left=1.5in,right=1.5in]{geometry}
\usepackage{polyglossia}
\newfontfamily\thaifont{TH Sarabun New}
\setmainfont{TH Sarabun New}
\setdefaultlanguage{thai}
\XeTeXlinebreaklocale 'th'
\usepackage{scrextend}
\changefontsizes[12pt]{12pt}
\XeTeXlinebreakskip = 0pt plus 1pt
\defaultfontfeatures{Scale=1.23}
\renewcommand{\baselinestretch}{1.2}

\usepackage{amsmath}

\begin{document}

หลังจากได้ฟังอาจารย์ให้คำอธิบายเรื่องข้อสอบกลางภาคแล้ว จึงได้กลับมาทบทวนตัวเองอีกครั้งครับ พบว่ามีเรื่องที่ยังคงไม่เข้าใจอีก 3 ส่วนด้วยกัน IR Model ที่ยังไม่เข้าใจถึงความสัมพันธ์ของสมการในโมเดลนั้นๆ วิธีการคำนวณ และการเปรียบเทียบประสิทธิภาพของโมเดล โดยได้ทำความเข้าใจเนื้อหาที่ยังไม่เข้าใจเพิ่มเติมมาดังนี้ครับ

\section{Information Retrieval Model}

\subsection{TF-IDF Weights}

การหา TF-IDF นั้นประกอบไปด้วย 2 ส่วนย่อยด้วยกันคือ Term Frequency Weights น้ำหนักที่คำนวนจากความถี่ของคำ (Term) ที่ปรากฎในเอกสารแต่ละชิ้น และ Inverse Document Frequency ที่นำมาถ่วงน้ำหนักเข้ากับความถี่ของคำที่พบในเอกสาร โดยมีรายการดังนี้

\subsubsection{TF - Term Frequency weights}

TF คือ น้ำหนักของคำใดๆ แยกตามเอกสาร โดยคิดจากความถี่ของคำที่เกิดขึ้นภายในเอกสารนั้นๆ โดยมีสูตรคือ
\begin{equation}
    \label{eq:tf}
    tf_{i,j} = 
    \begin{cases}
        1 + \log_2{f_{i,j}} &;  f_{i,j} > 0 \\
        0 & otherwise \\
    \end{cases}
\end{equation}

\subsubsection{IDF - Inverse Document Frequency}

\end{document}

